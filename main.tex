% DAFN24 - Robotics - Lecture 4
% Roberto Masocco <roberto.masocco@uniroma2.it>
% May 16, 2024

\documentclass[aspectratio=169]{beamer}

% Slides layout
\usepackage[
    title={ROS 2},
    subtitle={Advanced communication II},
    event={DAFN},
    author={Roberto Masocco},
    longauthor={Roberto Masocco},
    email={roberto.masocco@uniroma2.it},
    institute={Tor Vergata},
    longinstitute={University of Rome Tor Vergata},
    department={Department of Civil Engineering and Computer Science Engineering},
    researchgroup={},
    date={May 16, 2024}
]{utvengbeamer}

% Code listings settings
\usepackage[nomath]{lmodern}
\definecolor{codegreen}{rgb}{0 0.5 0}
\definecolor{codered}{rgb}{1 0 0}
\definecolor{codeocher}{rgb}{0.8 0.47 0.13}
\usepackage{listings}
\lstdefinestyle{beamer}{
    basicstyle=\ttfamily\small,
    commentstyle=\color{codegreen},
    breakatwhitespace=false,
    captionpos=b,
    frame=lines,
    keepspaces=true,
    keywordstyle=\color{codered}\bfseries,
    numbers=left,
    numbersep=5pt,
    numberstyle=\footnotesize,
    showspaces=false,
    showstringspaces=false,
    showtabs=false,
    stringstyle=\color{codeocher},
    tabsize=2
}
\lstset{style=beamer}
\lstdefinelanguage{ros2msg}{
  alsoletter={[, ], _, /},
  morecomment=[l][\color{codegreen}]{\#},
  morekeywords={int64, uint32, string, uint8, uint8[], int32, int32[], std_msgs/Header}
}

\usepackage{hyperref}
\usepackage{wasysym}

\begin{document}

% --- Title page ---
\frame{\titlepage}

% --- Recap ---
% Recap
% Roberto Masocco <roberto.masocco@uniroma2.it>
% May 16, 2024

% --- Recap ---
\begin{frame}{Recap}
\textbg{Messages} are the most basic, \textbg{one-way} communication paradigm.\\
\bigskip
\textbg{QoS policies} and other topic settings affect communication behavior among entities.\\
\bigskip
\textbg{Services} are a simple implementation of a \textbg{two-way}, \textbg{client-server} communication paradigm.\\
\bigskip
This lecture is \href{https://github.com/robmasocco/DAFN24_Robotics_4}{\color{blue}\underline{here}}.
\end{frame}
\begin{frame}{Recap}
  \begin{block}{Updates}
    \begin{itemize}
      \item \textbf{Updated code examples}, please pull new commits!
      \item Follow-up on \textbf{message topics code examples}:
      \begin{itemize}
        \item \href{https://github.com/IntelligentSystemsLabUTV/ros2-examples/blob/humble/src/cpp/topic_pubsub_cpp/src/resetting_sub.cpp}{\color{blue}\underline{\texttt{resetting\_sub}}} example.
      \end{itemize}
    \end{itemize}
  \end{block}
\end{frame}


% --- Table of contents ---
\begin{frame}
	\frametitle{Roadmap}
	\tableofcontents
\end{frame}

% --- Section 1 ---
% Section 1 - Interface packages
% Roberto Masocco <roberto.masocco@uniroma2.it>
% May 16, 2024

% ### Interface packages ###
\section{Interface packages}




% --- Section 2 ---
\input{sections/actions}

% --- Exercises ---
\begin{frame}{Exercises}
	\begin{itemize}
		\item Run the action client and server examples, and try to call the action from the command line.
		\item Modify the feedback message: instead of a partial sequence, it should publish the length of the sequence so far; this requires:
		      \begin{enumerate}
			      \item modifying the action definition in \texttt{ros2\_examples\_interfaces};
			      \item modifying the server node to publish the length of the sequence instead of the partial sequence in the feedback message (hint: use methods of the \texttt{std::vector} class to get the length of the partial sequence in one go);
			      \item modifying the feedback callback in the client to parse and print the length from the feedback message.
		      \end{enumerate}
	\end{itemize}
\end{frame}

\end{document}
