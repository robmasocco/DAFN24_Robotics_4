% Section 1 - Asynchronous I/O
% Roberto Masocco <roberto.masocco@uniroma2.it>
% May 24, 2023

% ### Asynchronous I/O ###
\section{Asynchronous I/O}
\graphicspath{{figs/section1/}}

% --- What is I/O? ---
\begin{frame}{What is I/O?}{Informal definition}
  In an \textbg{operating system}, a \textbg{task} (specifically, a \textbg{thread}) can perform operations pertaining to these two broad families:
  \begin{itemize}
    \item execute \textbg{computations} (\emph{e.g.}, \texttt{1 + 1 = 2}), using regular \textbg{CPU} instructions;
    \item access \textbg{system resources} (both \textbg{hardware} and \textbg{software}) through calls to the \textbg{kernel} (\emph{i.e.}, \textbg{system calls}), \textbg{exchanging data} in both directions.
  \end{itemize}
  When these resources are not part of the OS, but rather the OS enables tasks to interface\footnote{Drivers, protocols, software stacks...} with them, we talk about \textbg{I/O} (\emph{Input/Output}).
  \newline\newline
  OS schedulers typically distinguish between \textbg{CPU-bound} and \textbg{I/O-bound} tasks, because of their different \textbg{execution patterns}.
\end{frame}

% --- Blocking I/O ---
\begin{frame}{Blocking I/O}{What the OS likes the most}
  The most common execution pattern for a task that performs an I/O system call goes like this:
  \begin{enumerate}
    \item prepare the \textbg{input data} for the system call;
    \item call an \textbg{API} that performs the system call;
    \item the OS \textbg{blocks the task}, which is \textbg{waiting} for the operations to complete;
    \item the OS \textbg{returns control} to the task when the system call is completed;
    \item \textbg{output data}, returned by the kernel, can be accessed by the task.
  \end{enumerate}
  This is \textbg{blocking I/O}, because the task is \textbg{blocked} while waiting for the system call to complete.
  \newline\newline
  Examples of \textbg{blocking calls}: \texttt{read}, \texttt{write} to \textbg{file descriptors}.
\end{frame}

% --- Non-blocking I/O ---
\begin{frame}{Non-blocking I/O}{What userspace application like the most}
  If the kernel supports this feature, a task can perform a \textbg{non-blocking system call}:
  \begin{enumerate}
    \item prepare the \textbg{input data} for the system call;
    \item call an \textbg{API} that performs the system call;
    \item the OS \textbg{returns control} to the task \textbg{immediately}, without blocking it;
    \item the task can \textbg{poll} the system call \textbg{status} to check if it is completed;
    \item when the system call is completed, the task can \textbg{access the output data};
    \item optionally, a \textbg{callback} routine can be registered to be executed right when the system call is completed.
  \end{enumerate}
  This is \textbg{non-blocking I/O} (or \emph{asynchronous I/O}, or \emph{overlapped I/O}), because the task is \textbg{not blocked} while waiting for the system call to complete, and things can happen in between.
  \newline\newline
  Examples of \textbg{non-blocking calls}: \texttt{read}, \texttt{write} to \textbg{sockets} configured appropriately.
\end{frame}
\begin{frame}{Non-blocking I/O}{What userspace application like the most}
  Usually, the operation status can be inspected through some kind of \textbg{handle object} returned by the API.
  \newline\newline
  Some \textbg{programming languages} implement \textbg{future objects}: datatypes that hold the result of an asynchronous operation, which can be inspected to check if the operation is completed, and to retrieve the result once it is; \textbg{they are said to hold a value only when the operation is completed}.
  \begin{alertblock}{}
    \centering
    \textbr{ROS 2} makes a heavy use of \textbr{callbacks} and \textbr{future objects} to handle \textbr{asynchronous I/O}.
  \end{alertblock}
\end{frame}
